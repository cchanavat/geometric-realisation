\section{Directed simplex category}

\subsection{Combinatorics and lattice structure}

We call \( \Delta \) the simplex category whose objects are finite linear orders \( [n] = \{0 < \dots < n\} \) and morphisms (weakly) monotonic functions. It is generated by the face maps \( \delta_i : [n-1] \to [n] \) and the coface maps \( \sigma_j : [n+1] \to [n] \).

\begin{cdef}{Ordering}{ordering}
    An \cemph{ordering} \( P \) on \( [n] \in \Delta \) is a reflexive relation \( P \subseteq [n] \times [n]  \).
\end{cdef}

\begin{cdef}{Directed simplex}{directed_simplex}
    The \cemph{directed simplex} category \( \dDelta \) has object dependant pairs \( ([n], P \subseteq [n] \times [n]) \) with \( P \) an ordering, and morphisms are the one of \( \Delta \) that preserve orderings. More precisely, \( f : ([n], P) \to ([m], Q) \) is a morphism \( f : [n] \to [m] \) in \( \Delta \) such that \( f(P) \subseteq Q \), or equivalently, \( P \subseteq f^{-1}(Q) \).
\end{cdef}

We consider \( \Delta \) to be the subcategory of \( \dDelta \) by sending \( [n] \) to \( ([n], [n] \times [n]) \). As in \( \Delta \), we have the factorisation lemma of the maps.
\begin{clem}{Factorisation}{facto_maps}
    Any map \( f : ([n], P) \to ([m], Q) \) factorizes as 
    \begin{equation*}
        f = d_{i_1}\dots d_{i_p}s_{j_1}\dots s_{j_q}
    \end{equation*}
    where the orderings are constructed iteratively, starting from \( P \), that is:
    \begin{equation*}
        s_{j_k} : ([n], s_{j_{k+1}}\dots s_{j_q}(P)) \to ([n - k], s_{j_k}s_{j_{k+1}}\dots s_{j_q}(P)) 
    \end{equation*}
    and similarly for \( d_{i_k} \), except for \( d_{i_1} : ([n], d_{i_2}\dots d_{i_p}s_{j_1}\dots s_{j_q}(P)) \to ([m], Q) \) (or for \( s_{j_1} \) if \( p = 0 \)).
\end{clem} 
\begin{lemproof}{facto_maps}
    We use the decomposition from the underlying map \( f \) in \( \Delta \), and by construction, each \( s_{j_k}, d_{i_k} \) lift to \( \dDelta \).
\end{lemproof}
We exhibit a join semi-lattice structure on the objects of a fixed dimension, that in turn give set of sub-functors of a representable. We wall \( \Delta^{[n], P} \) the presheaf represented by \( ([n], P) \). It all boils down to the next lemma.
