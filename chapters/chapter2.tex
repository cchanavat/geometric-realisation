\section{Geometric realisation}

\subsection{Directed geometric realisation}

In this section, we describe how to realize a combinatorial computad (atoms + inclusions) into a directed topological space. The need to realize only the atoms, rather than all the category of diagrammatic sets comes from the absence of good functorial properties from the map we will describe below, in the presence of surjections between diagrams. Moreover, without the surjections, we consider the category \( \dDelta \) to be the directed semi-simplicial category, meaning that it is generated only by the maps \( \delta_i \) with the equation \( \delta_i\delta_j = \delta_{j+1}\delta_i \) whenever \( i \le j \). This is not too much of a restriction as Simon Henry's work on weak model structures shows that one can do homotopy theory just fine in this way. All the results of the previous section also apply.

We want to turn a directed semi-simplicial set into a directed topological space. For that, we realize the underlying semi-simplicial set the usual way, and we add a set of paths according to the ordering.  

\begin{cdef}{d-spaces and d-maps}{d_space}
    A \cemph{d-space} \( (X, dX) \) is a topological space together with a set \( dX \) of dipaths such that:
    \begin{enumerate}
        \item every constant path is a dipath,
        \item If \( \phi : [0, 1] \to [0, 1] \) is a directed partial reparametrisation (i.e. weakly increasing), then \( f \circ \phi \) is a dipath for all \( f \in dX \),
        \item The concatenation of two dipaths is a dipath. 
    \end{enumerate}
    A \cemph{d-map} (or dimap) \( f : (X, dX) \to (Y, dY) \) is a continuous map such that for all \( h \in dX \), \( f \circ h \in dY \). This gathers into the category \( \dTop \).
\end{cdef}

If \( x_i, x_j \) are points in \( \bbR \), we call \( \gamma_{ij} \) the path defined by \( \gamma_{ij}(t) = tx_i + (1-t)x_j \).

\begin{cdef}{Directed geometric realisation}{geom_real}
    The \cemph{directed geometric realisation} of an element \( ([n], P) \in \dDelta \) is the directed topological space \( (\Delta^n, dP) \) where \( \Delta^n \) is the convex hull spanned by \( n + 1 \) affine independent points \( x_0, \dots, x_n \) of \( \bbR^{n+1} \), and \( dP \) is the closure under concatenation and reparametrisation of the set of paths:
    \begin{equation*}
        \{ \sum_{(i,j) \in P} c_{ij} \gamma_{ij} \mid \sum_{(i,j) \in P} c_{ij} = 1 \}
    \end{equation*}
\end{cdef}
The directed geometric realisation has underlying topological space the standard geometric realisation. The set \( dP \) forms a set of dipaths. Indeed by definition it is closed under concatenation and reparametrisation. It also contains all the constant path, as if \( x \) is in \( \Delta^n \), then \( x = \sum_{i=0}^n c_ix_i \) with \( \sum_{i=0}^n c_i = 1 \), thus by setting \( c_{ij} = 0 \) if \( i \neq j \) and \( c_{ij} = c_i \) if \( i = j \), and by noticing that \( P \) is reflexive, we obtain the constant path at \( x \). To make it a functor \( |\dash|_d : \dDelta \to \dTop \), we only need to check that the underlying continuous maps from the standard simplex lifts to maps of directed topological space. Consider \( f : ([n], P) \to ([m], Q) \), and take \( \gamma = \sum_{(i,j) \in P} c_{ij} \gamma_{ij} \) in \( dP \). We want to show that \( |f|\gamma \) is in \( dQ \). By definition, \( |f| \) distributes over barycentric combinations, and \( |f|\gamma_{ij} = \gamma_{f(i)f(j)} \). For \( (i, j) \in Q \), set
\begin{equation*}
    c'_{ij} = \sum_{(k, l) \in f^{-1}(i, j)} c_{kl}
\end{equation*}
where \( f^{-1}(i, j) \) is shorthand for \( (f^{-1}(i)\times f^{-1}(j))\cap P \), and 
\begin{equation*}
    |f| \circ \gamma = \sum_{(k,l) \in P} c_{kl} \gamma_{f(k)f(l)} =  \sum_{(i,j) \in Q} c'_{ij} \gamma_{ij} 
\end{equation*}
with \( \sum_{(i,j) \in Q} c'_{ij} = \sum_{(k,l) \in P} c_{kl} = 1 \), as \( f(P) \subseteq Q \). Thus, \( |f| \circ \gamma \) is an admissible path of \( dQ \).

We can left-Kan extend this construction and obtain a functor \( |\dash|_{d\Delta} : \dsSet \to \dTop \).

\subsection{From combinatorial computad to directed simplicial sets}

We call \( \eye_i \) the skeleton of the category of combinatorial computad, that is atoms and inclusions. We have the following picture:
\begin{center}
    \begin{tikzcd}
        && \dDelta && \dTop \\
        \\
        \eye_i && \dsSet \\
        \\
        \eye_i\Sets
        \arrow["{|\dash|_{\dDelta}}", from=1-3, to=1-5]
        \arrow["\y"', hook, from=1-3, to=3-3]
        \arrow["\y"', hook, from=3-1, to=5-1]
        \arrow["{?}", dashed, from=3-1, to=3-3]
        \arrow["{\Lan_\y(?)}"', from=5-1, to=3-3]
        \arrow["{|\dash|_{d\Delta}}"', from=3-3, to=1-5]
    \end{tikzcd}
\end{center}
We would like to find the dashed functor, thus giving a geometric realization of a diagrammatic set. For that, recall from \cite{hadzihasanovic_diagrammatic_2020} that we have a functor:
\begin{equation*}
    k : \eye_i \to \sSet
\end{equation*} 
sending a molecule \( U \) to the simplicial nerve of its underlying poset. If \( U \) was already a simplex, it would correspond to constructing its barycentric subdivision. We would like to extend it to a functor \( k_d : \eye_i \to \dsSet \). As \( \Delta \subseteq \dDelta \), we already have a functor \( \sSet \to \dsSet \), however doing so would lose all the relevant information about the orientation of the diagram \( U \) in \( \eye_i \) that we try to convey. Thus, we need to refine the functor \( k \), by adding manually orientations. To that end, we introduce the intermediate category of preordered posets.

\begin{cdef}{Preordered poset}{preordered_poset}
    A \cemph{preordered poset} is a triple \( (P, \le, \preceq) \) where \( (P, \le) \) is a poset and \( (P, \preceq) \) is a preorder. A map \( f :  (P, \le, \preceq) \to  (Q, \le, \preceq) \) is a map respecting both the poset and the preorder structure. This form the category \( \prePos \).
\end{cdef}

We have a forgetful functor \( \prePos \to \Pos \) and another \( \eye_i \to \Pos \). We now describe how to factorize them via another forgetful functor from \( \U : \eye_i \to \prePos \). Let \( V \) be an atom, we define \( \U V \defeq (\H U, \le, \preceq)  \), where the posetal structure is the one of its Hasse diagram, and the preorder is the transitive and reflexive closure of the relation given by:
\begin{equation*}
    x \preceq y \iff o(x \leftarrow y) = - \text{ or } o(y \leftarrow x) = +
\end{equation*}   
As inclusions preserve orientation, this construction is functorial. We may now construct the directed nerve functor \( dN : \prePos \to \dsSet \). Let \( (X, \le, \preceq) \) in \( \prePos \). We define \( dN(X)([n], P) \) to be the set of \( (n+1) \) chains \( (x_0 \le \dots \le x_n ) \) (like in the ordinary nerve) that are moreover such that for all \( (i, j) \in P \), we have \( x_i \preceq x_j \). We call \( k_d : \eye_i \to \dsSet \) the functor \( dN \circ \U \).

We summarize our constructions with the following diagram where everything commutes. A dashed arrow indicates a forgetful functor.
\begin{center}
    \begin{tikzcd}
        \eye_i\Sets \arrow[rrrrrrd, "|\dash|_d \defeq \Lan_\y(|k_d\dash|_{d\Delta})", bend left]                                                                            &  &                                             &  &                                                              &  &                          \\
                    &  & \prePos \arrow[dd, dashed] \arrow[rr, "dN"] &  & \dsSet \arrow[dd, "\iota^*"] \arrow[rr, "|\dash|_{d\Delta}"] &  & \dTop \arrow[dd, dashed] \\
        \eye_i \arrow[rrrrd, "k"', bend right=49] \arrow[rrrru, "k_d", bend left=49] \arrow[rru, dashed] \arrow[rrd, dashed] \arrow[uu, "\y", hook] \arrow[dd, "\y"', hook] &  &                                             &  &                                                              &  &                          \\
        &  & \Pos \arrow[rr, "N"']                       &  & \sSet \arrow[rr, "|\dash|_\Delta"']                          &  & \Top                     \\
        \eye_i\Sets \arrow[rrrrrru, "|\dash|\defeq \Lan_\y(|k\dash|_\Delta)"',  bend right]    &  & &  &   &  &                         
        \end{tikzcd}
\end{center}

\subsection{From diagrammatic sets to directed simplicial sets}

In this part, we study a different version of the geometric realisation of diagrammatic sets. We will take the same approach as in the previous section, but we will have to deal with surjective maps between molecules, and thus we will need to take more care when constructing the forgetful functor to \( \prePos \). We notice that we can leave unchanged the \citedef{geom_real} on geometric realisation, as the proof of functoriality equally applies if we add degeneracies. What breaks is the construction of the preorder \( \preceq \), that does not behave functorialy as the surjections do not preserve orientations in general. Thus, we will adapt it here. 

We aim to construct a functor \( \U : \eye \to \prePos \). For that we consider \( V \in \eye \), and as usual, we associate it its underlying poset \( (\H V, \le) \). Now, the preorder is the reflexive and transitive closure of the relation given by:
\begin{equation*}
    x \preceq y \iff x \in \partial^-_0 y \text{ or } y \in \partial^+_0 x
\end{equation*}
That is whenever \( x \preceq y \) (and \( x \neq y \)), then \( x \) or \( y \) is a point (or both by transitivity). Now, a map \( f : P \to Q \) of \( \eye \) satisfies in particular 
\begin{equation*}
    \partial^\alpha_0f(x) = f(\partial^\alpha_0 x) 
\end{equation*}
for all \( x \in P \), thus if \(  x \in \partial^-_0 y \), then \(  f(x) \in \partial^-_0 f(y) \), making \( \U \) a functor \( \eye \to \prePos \). We now have the same picture as previously
\begin{center}
    \begin{tikzcd}
        \eye\Sets \arrow[rrrrrrd, "|\dash|_d \defeq \Lan_\y(|k_d\dash|_{d\Delta})", bend left]                                                                            &  &                                             &  &                                                              &  &                          \\
                    &  & \prePos \arrow[dd, dashed] \arrow[rr, "dN"] &  & \dsSet \arrow[dd, "\iota^*"] \arrow[rr, "|\dash|_{d\Delta}"] &  & \dTop \arrow[dd, dashed] \\
        \eye \arrow[rrrrd, "k"', bend right=49] \arrow[rrrru, "k_d", bend left=49] \arrow[rru, dashed] \arrow[rrd, dashed] \arrow[uu, "\y", hook] \arrow[dd, "\y"', hook] &  &                                             &  &                                                              &  &                          \\
        &  & \Pos \arrow[rr, "N"']                       &  & \sSet \arrow[rr, "|\dash|_\Delta"']                          &  & \Top                     \\
        \eye\Sets \arrow[rrrrrru, "|\dash|\defeq \Lan_\y(|k\dash|_\Delta)"',  bend right]    &  & &  &   &  &                         
        \end{tikzcd}
\end{center}
with the whole category of diagrammatic sets.