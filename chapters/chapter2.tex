\section{From diagrammatic sets to directed simplicial sets}

\subsection{Basic definitions}

We refer to \cite{hadzihasanovic_diagrammatic_2020} and \cite{hadzihasanovic_data_2022} for the basics of oriented graded posets and regular molecules.

\begin{cdef}{Diagrammatic sets}{diagrammatic_sets}
	A \cemph{diagrammatic set} is a presheaf on the category \( \eye \) whose
	\begin{itemize}
		\item objects are \emph{atoms} \( U \) of arbitrary dimension,
		\item morphisms \( f\colon U \to V \) are \emph{maps} of oriented graded posets, that is, functions satisfying
			\begin{equation*}
				f(\bound{n}{\alpha}x) = \bound{n}{\alpha}f(x)
			\end{equation*}
		for all \( x \in U, n \in \mathbb{N}, \alpha \in \{ +, - \} \).
	\end{itemize}
	Diagrammatic sets and their morphisms of presheaves form a category \( \dgmSet \).
\end{cdef}

\begin{cdef}{Combinatorial computads}{combinatorial_computads}
	A \cemph{combinatorial computad} is a presheaf on the wide subcategory \( \eyein \) of \( \eye \) whose morphisms are \emph{inclusions}, that is, injective maps.

	Combinatorial computads and their morphisms of presheaves form a category \( \CComp \).
\end{cdef}

We will define two realisation functors valued in directed simplicial sets:
\begin{enumerate}
	\item the first from diagrammatic sets,
	\item the second from combinatorial computads.
\end{enumerate}
In both cases, we will do this by first defining a functor from their site of definition, namely, \( \eye \) and \( \eyein \), respectively, then taking a left Kan extension along the Yoneda embedding.
Composing with \( \dirgeo{-} \) will then produce a realisation functor valued in d-spaces, as in the following picture.
\begin{center}
    \begin{tikzcd}
        && \dDelta && \dTop \\
        \\
        \eye && \dsSet \\
        \\
        \dgmSet
	\arrow["{\dirgeo{-}}", from=1-3, to=1-5]
        \arrow["\y"', hook, from=1-3, to=3-3]
        \arrow["\y"', hook, from=3-1, to=5-1]
        \arrow["{?}", dashed, from=3-1, to=3-3]
        \arrow["{\Lan_\y(?)}"', from=5-1, to=3-3]
	\arrow["{\dirgeo{-}}"', from=3-3, to=1-5]
    \end{tikzcd}
    \begin{tikzcd}
        && \dDelta && \dTop \\
        \\
        \eyein && \dsSet \\
        \\
        \CComp
	\arrow["{\dirgeo{-}}", from=1-3, to=1-5]
        \arrow["\y"', hook, from=1-3, to=3-3]
        \arrow["\y"', hook, from=3-1, to=5-1]
        \arrow["{?}", dashed, from=3-1, to=3-3]
        \arrow["{\Lan_\y(?)}"', from=5-1, to=3-3]
	\arrow["{\dirgeo{-}}"', from=3-3, to=1-5]
    \end{tikzcd}
\end{center}
Both of these will have the geometric realisation described in \cite[Section 8.3]{hadzihasanovic_diagrammatic_2020} as underlying functor.
Just as the definition of that functor goes through the simplicial nerve of posets, our definition will go through a ``directed'' version of this nerve.

\begin{cdef}{Direction on a poset}{direction_on_poset}
	A \cemph{direction on a poset} \( P \) is a reflexive relation \( \direc \) on the underlying set of \( P \).
	We let \( \dPos \) be the category of dependent pairs \( (P, \direc) \) where \( P \) is a poset and \( \direc \) a direction on \( P \), together with maps preserving both the order and the direction.
\end{cdef}

We have an obvious inclusion \( \dDelta \hookrightarrow \dPos \).

\begin{cdef}{Directed simplicial nerve of posets}{directed_simplicial_nerve}
	The \cemph{directed simplicial nerve of posets} is the right adjoint \( dN\colon \dPos \to \dsSet \) in the realisation-nerve pair induced by the inclusion \( \dDelta \hookrightarrow \dPos \).
\end{cdef}

Explicitly, given a poset with direction \( (P, \direcalt) \), an \( n \)-simplex with direction \( \direc \) in \( dN(P, \direcalt) \) is given by a chain
\begin{equation*}
	(x_0 \leq \ldots \leq x_n)
\end{equation*}
of length \( n+1 \) in \( P \) such that \( i \direc j \) implies \( x_i \direcalt x_j \) for all \( i, j \in [n] \).

To conclude, it will suffice to define a functor \( \eye \to \dPos \) or a functor \( \eyein \to \dPos \).


\subsection{Realisation of diagrammatic sets}

The realisation functor for diagrammatic sets will be induced by the functor
\begin{equation*}
	\U_1\colon \eye \to \dPos
\end{equation*}
sending an atom \( V \) to its underlying poset with the direction defined by
\begin{equation*}
	x \direc y \iff x = y \text{ or } x \in \bound{0}{-}y \text{ or } y \in \bound{0}{+}x.
\end{equation*}
This relation is reflexive by definition.
If \( f\colon U \to V \) is a map of atoms, it is proven in \cite[Lemma 1.9]{hadzihasanovic_diagrammatic_2020} that it is an order-preserving map of the underlying posets.
Moreover, by definition of maps, \( f(\bound{0}{\alpha}x) = \bound{0}{\alpha}f(x) \) for all \( x \in U, \alpha \in \{ +, - \} \), so \( x \direc y \) implies \( f(x) \direc f(y) \).
This proves that \( \U_1 \) is well-defined as a functor.

The following diagram summarises the construction of this directed geometric realisation and its relation to the ``undirected'' realisation.

\begin{center}
    \begin{tikzcd}
	    \dgmSet \arrow[rrrrrrd, "{\dirgeodgm{-} \defeq \Lan_\y(\dirgeo{k_1})}", bend left]                                                                            &  &                                             &  &                                                              &  &                          \\
																					&  & \dPos \arrow[dd, dashed] \arrow[rr, "dN"] &  & \dsSet \arrow[dd, "\iota^*"] \arrow[rr, "\dirgeo{-}"] &  & \dTop \arrow[dd, dashed] \\
        \eye \arrow[rrrrd, "k"', bend right=49] \arrow[rrrru, "k_1", bend left=49] \arrow[rru, "\U_1"] \arrow[rrd, dashed] \arrow[uu, "\y", hook] \arrow[dd, "\y"', hook] &  &                                             &  &                                                              &  &                          \\
        &  & \Pos \arrow[rr, "N"']                       &  & \sSet \arrow[rr, "|\dash|_\Delta"']                          &  & \Top                     \\
        \dgmSet \arrow[rrrrrru, "|\dash|\defeq \Lan_\y(|k\dash|_\Delta)"',  bend right]    &  & &  &   &  &                         
        \end{tikzcd}
\end{center}

\subsection{Realisation of combinatorial computads}

The realisation defined in the previous section only cares about ``direction in dimension 1'': reversing \( n \)-cells for any \( n > 1 \) results in the same d-space up to isomorphism.

In this section we will define a realisation functor which also encodes higher-dimensional directions via the existence of dipaths.
However this realisation is incompatible with the presence of \emph{collapsing} maps of atoms, as it is possible, in diagrammatic sets, to collapse a pair of cells one of which is in the input boundary of the other to a pair of lower-dimensional cells one of which is in the output boundary of the other.
This produces an illegal ``inversion'' if directions in all dimensions are encoded via directed paths.

For this reason, this realisation only extends functorially to \emph{inclusions} of atoms, hence determining a functor on combinatorial computads rather than diagrammatic sets.

\begin{cdef}{Oriented Hasse diagram}{oriented_hasse}
Let \( U \) be an atom.
The \cemph{oriented Hasse diagram of \( U \)} is the directed graph \( \hasseo{U} \) whose
\begin{itemize}
    \item set of vertices is the underlying set of \( U \), and
    \item for all vertices \( x, y \), there is an edge from \( y \) to \( x \) if and only if \( y \in \faces{}{-}x \) or \( x \in \faces{}{+}y \).
\end{itemize}
\end{cdef}

The realisation functor for combinatorial computads will be induced by the functor
\begin{equation*}
	\U_\infty\colon \eyein \to \dPos
\end{equation*}
sending an atom \( V \) to its underlying poset with the direction defined by
\begin{equation*}
	x \direc y \iff 
	\text{ there is an upward or downward path from \( x \) to \( y \) in \( \hasseo{U} \), }
\end{equation*}
where by upward and downward we mean ``increasing or decreasing in dimension'' (``zig-zag'' paths are not allowed).

This relation is clearly reflexive.
By \cite[Lemma 1.11]{hadzihasanovic_diagrammatic_2020}, inclusions of atoms preserve dimensions, orientations, and the covering relation, so they induce embeddings of oriented Hasse diagrams.
This implies that \( \U_\infty \) is well-defined as a functor.

The following diagram summarises this alternative directed realisation.

\begin{center}
    \begin{tikzcd}
	    \CComp \arrow[rrrrrrd, "{\dirgeocomp{-} \defeq \Lan_\y(\dirgeo{k_\infty})}", bend left]                                                                            &  &                                             &  &                                                              &  &                          \\
																					&  & \dPos \arrow[dd, dashed] \arrow[rr, "dN"] &  & \dsSet \arrow[dd, "\iota^*"] \arrow[rr, "\dirgeo{-}"] &  & \dTop \arrow[dd, dashed] \\
        \eyein \arrow[rrrrd, "k"', bend right=49] \arrow[rrrru, "k_\infty", bend left=49] \arrow[rru, "\U_\infty"] \arrow[rrd, dashed] \arrow[uu, "\y", hook] \arrow[dd, "\y"', hook] &  &                                             &  &                                                              &  &                          \\
        &  & \Pos \arrow[rr, "N"']                       &  & \sSet \arrow[rr, "|\dash|_\Delta"']                          &  & \Top                     \\
        \CComp \arrow[rrrrrru, "|\dash|\defeq \Lan_\y(|k\dash|_\Delta)"',  bend right]    &  & &  &   &  &                         
        \end{tikzcd}
\end{center}

We note that a simpler alternative to this construction would be to take the \emph{(directed) reachability relation} on the oriented Hasse diagram of an atom (that is, also allow ``zig-zag'' paths).
However, this has the consequence that, whenever there is a loop in the oriented Hasse diagram, which tends to happen from dimension 3 onward, every 1-simplex on the path traced by this loop is made reversible, which seems to produce a somewhat less interesting result.
