\section{From diagrammatic sets to directed simplicial sets}

\subsection{Basic definitions}

We refer to \cite{hadzihasanovic_diagrammatic_2020} and \cite{hadzihasanovic_data_2022} for the basics of oriented graded posets and regular molecules.

\begin{cdef}{Diagrammatic sets}{diagrammatic_sets}
	A \cemph{diagrammatic set} is a presheaf on the category \( \eye \) whose
	\begin{itemize}
		\item objects are \emph{atoms} \( U \) of arbitrary dimension,
		\item morphisms \( f\colon U \to V \) are \emph{maps} of oriented graded posets, that is, functions satisfying
			\begin{equation*}
				f(\bound{n}{\alpha}x) = \bound{n}{\alpha}f(x)
			\end{equation*}
		for all \( x \in U, n \in \mathbb{N}, \alpha \in \{ +, - \} \).
	\end{itemize}
	Diagrammatic sets and their morphisms of presheaves form a category \( \dgmSet \).
\end{cdef}

\begin{cdef}{Combinatorial computads}{combinatorial_computads}
	A \cemph{combinatorial computad} is a presheaf on the wide subcategory \( \eyein \) of \( \eye \) whose morphisms are \emph{inclusions}, that is, injective maps.

	Combinatorial computads and their morphisms of presheaves form a category \( \CComp \).
\end{cdef}

We will define two realisation functors valued in directed simplicial sets:
\begin{enumerate}
	\item the first from diagrammatic sets,
	\item the second from combinatorial computads.
\end{enumerate}
In both cases, we will do this by first defining a functor from their site of definition, namely, \( \eye \) and \( \eyein \), respectively, then taking a left Kan extension along the Yoneda embedding.
Composing with \( \dirgeo{-} \) will then produce a realisation functor valued in d-spaces, as in the following picture.
\begin{center}
    \begin{tikzcd}
        && \dDelta && \dTop \\
        \\
        \eye && \dsSet \\
        \\
        \dgmSet
	\arrow["{\dirgeo{-}}", from=1-3, to=1-5]
        \arrow["\y"', hook, from=1-3, to=3-3]
        \arrow["\y"', hook, from=3-1, to=5-1]
        \arrow["{?}", dashed, from=3-1, to=3-3]
        \arrow["{\Lan_\y(?)}"', from=5-1, to=3-3]
	\arrow["{\dirgeo{-}}"', from=3-3, to=1-5]
    \end{tikzcd}
    \begin{tikzcd}
        && \dDelta && \dTop \\
        \\
        \eyein && \dsSet \\
        \\
        \CComp
	\arrow["{\dirgeo{-}}", from=1-3, to=1-5]
        \arrow["\y"', hook, from=1-3, to=3-3]
        \arrow["\y"', hook, from=3-1, to=5-1]
        \arrow["{?}", dashed, from=3-1, to=3-3]
        \arrow["{\Lan_\y(?)}"', from=5-1, to=3-3]
	\arrow["{\dirgeo{-}}"', from=3-3, to=1-5]
    \end{tikzcd}
\end{center}
Both of these will have the geometric realisation described in \cite[Section 8.3]{hadzihasanovic_diagrammatic_2020} as underlying functor.

\begin{cdef}{Direction on a poset}{direction_on_poset}
	A \emph{direction} on a poset \( P \) is a reflexive relation \( \direc \) on the underlying set of \( P \).
	We let \( \dPos \) be the category of dependent pairs \( (P, \direc) \) where \( P \) is a poset and \( \direc \) a direction on \( P \), together with maps preserving both the order and the direction.
\end{cdef}

\subsection{Realisation of diagrammatic sets}

In this part, we study a different version of the geometric realisation of diagrammatic sets. We will take the same approach as in the previous section, but we will have to deal with surjective maps between molecules, and thus we will need to take more care when constructing the forgetful functor to \( \prePos \). We notice that we can leave unchanged the \citedef{geom_real} on geometric realisation, as the proof of functoriality equally applies if we add degeneracies. What breaks is the construction of the preorder \( \preceq \), that does not behave functorialy as the surjections do not preserve orientations in general. Thus, we will adapt it here. 

We aim to construct a functor \( \U : \eye \to \prePos \). For that we consider \( V \in \eye \), and as usual, we associate it its underlying poset \( (\H V, \le) \). Now, the preorder is the reflexive and transitive closure of the relation given by:
\begin{equation*}
    x \preceq y \iff x \in \partial^-_0 y \text{ or } y \in \partial^+_0 x
\end{equation*}
That is whenever \( x \preceq y \) (and \( x \neq y \)), then \( x \) or \( y \) is a point (or both by transitivity). Now, a map \( f : P \to Q \) of \( \eye \) satisfies in particular 
\begin{equation*}
    \partial^\alpha_0f(x) = f(\partial^\alpha_0 x) 
\end{equation*}
for all \( x \in P \), thus if \(  x \in \partial^-_0 y \), then \(  f(x) \in \partial^-_0 f(y) \), making \( \U \) a functor \( \eye \to \prePos \). We now have the same picture as previously
\begin{center}
    \begin{tikzcd}
        \eye\Sets \arrow[rrrrrrd, "|\dash|_d \defeq \Lan_\y(|k_d\dash|_{d\Delta})", bend left]                                                                            &  &                                             &  &                                                              &  &                          \\
                    &  & \prePos \arrow[dd, dashed] \arrow[rr, "dN"] &  & \dsSet \arrow[dd, "\iota^*"] \arrow[rr, "|\dash|_{d\Delta}"] &  & \dTop \arrow[dd, dashed] \\
        \eye \arrow[rrrrd, "k"', bend right=49] \arrow[rrrru, "k_d", bend left=49] \arrow[rru, dashed] \arrow[rrd, dashed] \arrow[uu, "\y", hook] \arrow[dd, "\y"', hook] &  &                                             &  &                                                              &  &                          \\
        &  & \Pos \arrow[rr, "N"']                       &  & \sSet \arrow[rr, "|\dash|_\Delta"']                          &  & \Top                     \\
        \eye\Sets \arrow[rrrrrru, "|\dash|\defeq \Lan_\y(|k\dash|_\Delta)"',  bend right]    &  & &  &   &  &                         
        \end{tikzcd}
\end{center}
with the whole category of diagrammatic sets.
\subsection{Realisation of combinatorial computads}

We call \( \eye_i \) the skeleton of the category of combinatorial computad, that is atoms and inclusions. We have the following picture:
\begin{center}
    \begin{tikzcd}
        && \dDelta && \dTop \\
        \\
        \eye_i && \dsSet \\
        \\
        \eye_i\Sets
        \arrow["{|\dash|_{\dDelta}}", from=1-3, to=1-5]
        \arrow["\y"', hook, from=1-3, to=3-3]
        \arrow["\y"', hook, from=3-1, to=5-1]
        \arrow["{?}", dashed, from=3-1, to=3-3]
        \arrow["{\Lan_\y(?)}"', from=5-1, to=3-3]
        \arrow["{|\dash|_{d\Delta}}"', from=3-3, to=1-5]
    \end{tikzcd}
\end{center}
We would like to find the dashed functor, thus giving a geometric realization of a diagrammatic set. For that, recall from \cite{hadzihasanovic_diagrammatic_2020} that we have a functor:
\begin{equation*}
    k : \eye_i \to \sSet
\end{equation*} 
sending a molecule \( U \) to the simplicial nerve of its underlying poset. If \( U \) was already a simplex, it would correspond to constructing its barycentric subdivision. We would like to extend it to a functor \( k_d : \eye_i \to \dsSet \). As \( \Delta \subseteq \dDelta \), we already have a functor \( \sSet \to \dsSet \), however doing so would lose all the relevant information about the orientation of the diagram \( U \) in \( \eye_i \) that we try to convey. Thus, we need to refine the functor \( k \), by adding manually orientations. To that end, we introduce the intermediate category of preordered posets.

\begin{cdef}{Preordered poset}{preordered_poset}
    A \cemph{preordered poset} is a triple \( (P, \le, \preceq) \) where \( (P, \le) \) is a poset and \( (P, \preceq) \) is a preorder. A map \( f :  (P, \le, \preceq) \to  (Q, \le, \preceq) \) is a map respecting both the poset and the preorder structure. This form the category \( \prePos \).
\end{cdef}

We have a forgetful functor \( \prePos \to \Pos \) and another \( \eye_i \to \Pos \). We now describe how to factorize them via another forgetful functor from \( \U : \eye_i \to \prePos \). Let \( V \) be an atom, we define \( \U V \defeq (\H U, \le, \preceq)  \), where the posetal structure is the one of its Hasse diagram, and the preorder is the transitive and reflexive closure of the relation given by:
\begin{equation*}
    x \preceq y \iff o(x \leftarrow y) = - \text{ or } o(y \leftarrow x) = +
\end{equation*}   
As inclusions preserve orientation, this construction is functorial. We may now construct the directed nerve functor \( dN : \prePos \to \dsSet \). Let \( (X, \le, \preceq) \) in \( \prePos \). We define \( dN(X)([n], P) \) to be the set of \( (n+1) \) chains \( (x_0 \le \dots \le x_n ) \) (like in the ordinary nerve) that are moreover such that for all \( (i, j) \in P \), we have \( x_i \preceq x_j \). We call \( k_d : \eye_i \to \dsSet \) the functor \( dN \circ \U \).

We summarize our constructions with the following diagram where everything commutes. A dashed arrow indicates a forgetful functor.
\begin{center}
    \begin{tikzcd}
        \eye_i\Sets \arrow[rrrrrrd, "|\dash|_d \defeq \Lan_\y(|k_d\dash|_{d\Delta})", bend left]                                                                            &  &                                             &  &                                                              &  &                          \\
                    &  & \prePos \arrow[dd, dashed] \arrow[rr, "dN"] &  & \dsSet \arrow[dd, "\iota^*"] \arrow[rr, "|\dash|_{d\Delta}"] &  & \dTop \arrow[dd, dashed] \\
        \eye_i \arrow[rrrrd, "k"', bend right=49] \arrow[rrrru, "k_d", bend left=49] \arrow[rru, dashed] \arrow[rrd, dashed] \arrow[uu, "\y", hook] \arrow[dd, "\y"', hook] &  &                                             &  &                                                              &  &                          \\
        &  & \Pos \arrow[rr, "N"']                       &  & \sSet \arrow[rr, "|\dash|_\Delta"']                          &  & \Top                     \\
        \eye_i\Sets \arrow[rrrrrru, "|\dash|\defeq \Lan_\y(|k\dash|_\Delta)"',  bend right]    &  & &  &   &  &                         
        \end{tikzcd}
\end{center}



